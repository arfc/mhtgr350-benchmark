\documentclass[11pt,letterpaper]{article}

\addtolength{\oddsidemargin}{-.875in}
\addtolength{\evensidemargin}{-.875in}
\addtolength{\textwidth}{1.75in}

\addtolength{\topmargin}{-.875in}
\addtolength{\textheight}{1.75in}

\usepackage[utf8]{inputenc}
\usepackage{caption} % for table captions
\usepackage{amsmath} % for multi-line equations and piecewises
\DeclareMathOperator{\sign}{sign}
\usepackage{graphicx}
\usepackage{relsize}
\usepackage{xspace}
\usepackage{verbatim} % for block comments
\usepackage{subcaption} % for subfigures
\usepackage{enumitem} % for a) b) c) lists
\newcommand{\Cyclus}{\textsc{Cyclus}\xspace}%
\newcommand{\Cycamore}{\textsc{Cycamore}\xspace}%
\newcommand{\deploy}{\texttt{d3ploy}\xspace}%
\newcommand{\Deploy}{\texttt{D3ploy}\xspace}%
\usepackage{tabularx}
\usepackage{color}
\usepackage{multirow}
\usepackage{float} 
\usepackage[acronym,toc]{glossaries}
%\include{acros}
\definecolor{bg}{rgb}{0.95,0.95,0.95}
\newcolumntype{b}{X}
\newcolumntype{f}{>{\hsize=.15\hsize}X}
\newcolumntype{s}{>{\hsize=.5\hsize}X}
\newcolumntype{m}{>{\hsize=.75\hsize}X}
\newcolumntype{r}{>{\hsize=1.1\hsize}X}
\usepackage{titling}
\usepackage[hang,flushmargin]{footmisc}
\renewcommand*\footnoterule{}
\usepackage{tikz}

\usetikzlibrary{shapes.geometric,arrows}
\tikzstyle{process} = [rectangle, rounded corners, 
minimum width=1cm, minimum height=1cm,text centered, draw=black, 
fill=blue!30]
\tikzstyle{arrow} = [thick,->,>=stealth]

\graphicspath{}
% \title{MHTGR350}
%\author{Roberto E. Fairhurst Agosta}

\begin{document}
%	\begin{titlepage}
%		\maketitle
%		\thispagestyle{empty}
%	\end{titlepage}

\section{Fuel Compact}

	\begin{table}[]
		\centering
	    \caption{TRISO and Fuel Compact Characteristics.}
	    \label{tab:compact}
		\begin{tabular}{l|l}
		\hline
		Characteristic                   & Value                \\ \hline
		Fuel                             & UC$_{0.5}$O$_{1.5}$  \\
		Enrichment (average)             & 15.5 wt\%            \\
		Kernel radius                    & 0.02125 cm           \\
		Buffer radius                    & 0.03125 cm           \\
		IPyC radius                      & 0.03475 cm           \\
		SiC radius                       & 0.03825 cm           \\
		OPyC radius                      & 0.04225 cm           \\
    	Kernel density                   & 10.5 g/cm$^3$        \\
		Buffer density                   & 1.0 g/cm$^3$         \\
		IPyC density                     & 1.9 g/cm$^3$         \\
		SiC density                      & 3.2 g/cm$^3$         \\
		OPyC density                     & 1.9 g/cm$^3$         \\
		Packing Fraction (average)       & 0.35                 \\
		Compact radius                   & 0.6223 cm            \\
		Compact Gap radius               & 0.635 cm             \\
		Compact length                   & 4.928 cm             \\ 
        Helium density           		 & 4.19 kg/m$^3$        \\
        Block graphite density           & 1.85 g/cm$^3$        \\ \hline

		\end{tabular}
	\end{table}

\section{Fuel Assembly}

	\begin{table}[H]
		\centering
	    \caption{MHTGR350 standard fuel element characteristics.}
	    \label{tab:scharacteristics}
		\begin{tabular}{l|l}
		\hline
		Characteristic                   & Value         \\ \hline
		Block pitch (flat-to-flat)       & 36 cm         \\
		Block graphite density           & 1.85 g/cm$^3$ \\
		Number of fuel holes             & 210           \\
		Fuel hole radius                 & 0.635 cm      \\
		Compacts per hole                & 15            \\
		Number of large coolant holes    & 120           \\
		Large coolant hole radius        & 0.794 cm      \\
		Number of small coolant holes    & 6             \\
		Small coolant hole radius        & 0.635 cm      \\
		Fuel/coolant pitch               & 1.879 cm      \\
		Fuel handling diameter           & 3.5 cm        \\ 
		Fuel handling length             & 26.4 cm       \\ 
		Fuel length                      & 79.3 cm       \\ \hline
		\end{tabular}
	\end{table}

	\begin{table}[H]
		\centering
	    \caption{MHTGR350 RSC fuel element characteristics.}
	    \label{tab:rcharacteristics}
		\begin{tabular}{l|l}
		\hline
		Characteristic                   & Value         \\ \hline
		Block pitch (flat-to-flat)       & 36 cm         \\
		Block graphite density           & 1.85 g/cm$^3$ \\
		Number of Fuel holes             & 186           \\
		Fuel hole radius                 & 0.635 cm      \\
		Compacts per hole                & 15            \\
		Number of large coolant holes    & 88            \\
		Large coolant hole radius        & 0.794 cm      \\
		Number of small coolant holes    & 7             \\
		Small coolant hole radius        & 0.635 cm      \\
		Fuel/coolant pitch               & 1.879 cm      \\ 
		RSC hole diameter                & 9.525 cm      \\
		RSC center to assembly center    & 9.756 cm      \\
		Fuel handling diameter           & 3.5 cm        \\ 
		Fuel handling length             & 26.4 cm       \\
		Fuel length                      & 79.3 cm       \\ \hline
		\end{tabular}
	\end{table}

\section{Full Core}

Table \ref{tab:ccharacteristics} specifies the characteristics of the core.

The model in serpent considers the following simplifications:
\begin{itemize}
	\item all the fuel elements are standard.
	\item reactor diameter: 6.6 m.
    \item control rods are not modeled.
	\item inner and outer reflector blocks are solid.
\end{itemize}

	\begin{table}[H]
		\centering
	    \caption{MHTGR350 Core Characteristics.}
	    \label{tab:ccharacteristics}
		\begin{tabular}{l|l}
		\hline
		Core Parameter                   & Value     \\ \hline
		Number of fuel columns           & 66        \\
		Number of standard fuel columns  & 54        \\
		Number of RSC fuel columns       & 12        \\
		Number of standard fuel elements & 540       \\
		Number of RSC fuel elements      & 120       \\
		Reactor vessel inner diameter    & 6.55 m    \\
		Height of the Core               & 7.93 m    \\
		Bottom reflector height          & 1.6 m     \\
		Top reflector height             & 1.2 m     \\ \hline
		\end{tabular}
	\end{table}

\pagebreak
\bibliographystyle{plain}
\bibliography{bibliography}

\end{document}
