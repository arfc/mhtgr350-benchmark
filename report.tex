\documentclass[11pt,letterpaper]{article}

\addtolength{\oddsidemargin}{-.875in}
\addtolength{\evensidemargin}{-.875in}
\addtolength{\textwidth}{1.75in}

\addtolength{\topmargin}{-.875in}
\addtolength{\textheight}{1.75in}

\usepackage[utf8]{inputenc}
\usepackage{caption} % for table captions
\usepackage{amsmath} % for multi-line equations and piecewises
\DeclareMathOperator{\sign}{sign}
\usepackage{graphicx}
\usepackage{relsize}
\usepackage{xspace}
\usepackage{verbatim} % for block comments
\usepackage{subcaption} % for subfigures
\usepackage{enumitem} % for a) b) c) lists
\newcommand{\Cyclus}{\textsc{Cyclus}\xspace}%
\newcommand{\Cycamore}{\textsc{Cycamore}\xspace}%
\newcommand{\deploy}{\texttt{d3ploy}\xspace}%
\newcommand{\Deploy}{\texttt{D3ploy}\xspace}%
\usepackage{tabularx}
\usepackage{color}
\usepackage{multirow}
\usepackage{float} 
\usepackage[acronym,toc]{glossaries}
%\include{acros}
\definecolor{bg}{rgb}{0.95,0.95,0.95}
\newcolumntype{b}{X}
\newcolumntype{f}{>{\hsize=.15\hsize}X}
\newcolumntype{s}{>{\hsize=.5\hsize}X}
\newcolumntype{m}{>{\hsize=.75\hsize}X}
\newcolumntype{r}{>{\hsize=1.1\hsize}X}
\usepackage{titling}
\usepackage[hang,flushmargin]{footmisc}
\renewcommand*\footnoterule{}
\usepackage{tikz}

\usetikzlibrary{shapes.geometric,arrows}
\tikzstyle{process} = [rectangle, rounded corners, 
minimum width=1cm, minimum height=1cm,text centered, draw=black, 
fill=blue!30]
\tikzstyle{arrow} = [thick,->,>=stealth]

\graphicspath{}
% \title{MHTGR350}
%\author{Roberto E. Fairhurst Agosta}

\begin{document}
%	\begin{titlepage}
%		\maketitle
%		\thispagestyle{empty}
%	\end{titlepage}

\section{Introduction}

In 2013, the IAEA launched the Coordinated Research Project (CRP) on Uncertainty Analysis in Modeling (UAM) to study uncertainty propagation in the High Temperature Gas-cooled Reactor (HTGR) analysis chain.

HTGR reactors require core simulation techniques not typically utilized in Light Water Reactor (LWR) analysis due to several unique features, such as double heterogeneous fuel design including tristructural isotropic (TRISO) fuel particles, large graphite quantities, and high operational temperatures.

\cite{bostelmann_criticality_2016}.

\section{Exercise I-1a}

Objective is to address the uncertainties due to double heterogeneity or self-shielding treatment.

Figure \ref{fig:compact} shows the MHTGR fuel compact unit cell.
This unit cell is derived from the fuel block hexagonal geometry, Figure \ref{fig:fuelblock}.

Exercise I-1a specifies a homogeneous fuel region.
Exercise I-1b specifies the TRISO fuel particles explicitly.

The problem uses a reflective boundary condition.
Two sub-cases: a Cold Zero Power (CZP) and Hot Full Power (HFP)
We will focus on the first one.

Differences to OECD MHTGR Benchmark.

	\begin{figure}[htbp!]
		\centering
		\begin{subfigure}[t]{0.4\textwidth}
			\centering
			\includegraphics[width=\linewidth]{exerciseI-1-geo}
			\caption{MHTGR fuel compact unit cell for Exercise I-1.}
			\label{fig:compact}
		\end{subfigure}
		\begin{subfigure}[t]{0.4\textwidth}
			\centering
			\includegraphics[width=\linewidth]{exerciseI-2-geo}
			\caption{MHTGR fuel block for Exercise I-2.}
			\label{fig:fuelblock}
		\end{subfigure}
		\hfill
		\caption{MHTGR fuel compact unit cell for Exercise I-1. Image reproduced from \cite{strydom_results_2015}.}
		\label{fig:fuel}
	\end{figure}






\pagebreak
\bibliographystyle{plain}
\bibliography{bibliography}

\end{document}

	% \begin{table}[]
	% 	\centering
	%     \caption{TRISO and Fuel Compact Characteristics.}
	%     \label{tab:compact}
	% 	\begin{tabular}{l|l}
	% 	\hline
	% 	Characteristic                   & Value                \\ \hline
	% 	Fuel                             & UC$_{0.5}$O$_{1.5}$  \\
	% 	Enrichment (average)             & 15.5 wt\%            \\
	% 	Kernel radius                    & 0.02125 cm           \\
	% 	Buffer radius                    & 0.03125 cm           \\
	% 	IPyC radius                      & 0.03475 cm           \\
	% 	SiC radius                       & 0.03825 cm           \\
	% 	OPyC radius                      & 0.04225 cm           \\
	%  	Kernel density                   & 10.5 g/cm$^3$        \\
	% 	Buffer density                   & 1.0 g/cm$^3$         \\
	% 	IPyC density                     & 1.9 g/cm$^3$         \\
	% 	SiC density                      & 3.2 g/cm$^3$         \\
	% 	OPyC density                     & 1.9 g/cm$^3$         \\
	% 	Packing Fraction (average)       & 0.35                 \\
	% 	Compact radius                   & 0.6223 cm            \\
	% 	Compact Gap radius               & 0.635 cm             \\
	% 	Compact length                   & 4.928 cm             \\ 
	%   Helium density           		 & 4.19 kg/m$^3$        \\
	%   Block graphite density           & 1.85 g/cm$^3$        \\ \hline

	% 	\end{tabular}
	% \end{table}

	% \begin{figure}[htbp!]
	% 	\centering
	% 	\begin{subfigure}[t]{0.4\textwidth}
	% 		\centering
	% 		\includegraphics[width=\linewidth]{figures/standard.png}
	% 		\caption{XY-plane.}
	% 	\end{subfigure}
	% 	\begin{subfigure}[t]{0.4\textwidth}
	% 		\centering
	% 		\includegraphics[width=\linewidth]{figures/standard-column.png}
	% 		\caption{YZ-plane.}
	% 	\end{subfigure}
	% 	\hfill
	% 	\caption{Plot of \textit{standard-column}.}
	% 	\label{fig:standardcolumn}
	% \end{figure}

	% \begin{figure}[htbp!]
	% 	\centering
	% 	\includegraphics[height=5cm]{figures/standard.png}
	% 	\caption{Standard fuel assembly model.}
	% 	\label{fig:standard}
	% \end{figure}